\documentclass[sigconf]{acmart}

\usepackage{hyperref}

\usepackage{endfloat}
\renewcommand{\efloatseparator}{\mbox{}} % no new page between figures

\usepackage{booktabs} % For formal tables

\settopmatter{printacmref=false} % Removes citation information below abstract
\renewcommand\footnotetextcopyrightpermission[1]{} % removes footnote with conference information in first column
\pagestyle{plain} % removes running headers

\begin{Paper1}
\title{Big Data Applications in Improving Patient Care}


\author{Janaki Mudvari Khatiwada}
\orcid{}
\affiliation{%
  \institution{University of Indiana}
  \streetaddress{}
  \city{Bloomington} 
  \state{Indiana} 
  \postcode{47408}
}
\email{jmudvari@iu.edu}

% The default list of authors is too long for headers}
\renewcommand{\shortauthors}{B. Trovato et al.}


\begin{abstract}
This paper will explore how service providers in health-care indus-tries use data generated when patients provide information about their family history, medical history, food habit, exercise habit.
\end{abstract}

\keywords{}


\maketitle

\section{Introduction}
Health service providers collect high volume of information from the consumers every time they visit the facilities.
These informations or big data provides helpful insights for diagnostic purpose and treatment options. These data can range from Clinical or pathological category to food and exercise habits, family history or personal body mass index.
Clinical practitioners require data to make their medical diagnosis, treatment
recommendation, and prognosis. A richer set of near-real-time information can greatly help
physicians determine the best course of action for their patients, discover new treatment
options, and potentially save lives~\cite{www.hpe}. So to speak fields big data applications in health care for the purpose of improving patient care is wide; disease prevention and management, health education, research and development, prognosis
information sharing, public and individual health management, medical optimization.

Health data are stored as electronic medical records(EMR) which are analyzed and shared among clinicians. These data
are near real time data. One of the trending example is application of big data in tackling opioid crisis in US.
Data scientists at Blue Cross Blue Shield have started working with big data experts at Fuzzy Logix to tackle the problem. Using years of insurance and pharmacy data, Fuzzy Logix analysts have been able to identify 742 risk factors that predict with a high degree of accuracy whether someone is at risk for abusing opioids\cite{Lebied:Example}.

\begin{acks}

  The authors would like to thank 

\end{acks}

\bibliographystyle{ACM-Reference-Format}
\bibliography{report} 

\end{document}
